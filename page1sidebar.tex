\cvsection{Career Objective}
\begin{quote}
``Improve and develop knowledge seeking alignment between the company's development and my professional growth.''
\end{quote}



\cvsection{Strengths}

\cvtag{Analytical thinking} 
\cvtag{Problem-solver}
\cvtag{Motivator}
\cvtag{Team worker}

\divider\smallskip

\cvtag{Geophysics}
\cvtag{Statistics}
\cvtag{Data analysis}


\cvsection{Languages}

\cvskill{Portuguese}{5}
% \divider

\cvskill{English}{4}
% \divider

\cvskill{Spanish}{2}
% \divider


\cvsection{SOFTWARE \& programming}

\cvtag{Python}
\cvtag{R}
\cvtag{Matlab}
\cvtag{SonarWiz}
\cvtag{CARIS}
\cvtag{SeiSee}
\cvtag{LateX}


\cvsection{ADDITIONAL INFORMATION}

 \faBook \space  Courses
  \begin{itemize}
      \item Multi-channel Analysis of Surface Wave (MASW) | Short course | Institution: USP, SP, Brazil
      \item Processing and interpretation techniques of unconsolidated core |Short course| SBGGM, Brazil
      \item Python Fundamentals for Data Analysis| Data Science Academy 
  \end{itemize}
  \divider
 
  \faGraduationCap \pace \sapce  Member of Geoscientific Student Society 
    \begin{itemize}
        \item Student Geophysics Chapter - UFF :SEG - Society of Exploration Geophysicists and EAGE - European Association of Geoscientists and Engineers
    \end{itemize}
    
\divider
    
    
 \faCar \space \space Driver’s License B


    \divider
  
  \faHandPeaceO \space \space Recreation
  \begin{itemize}
        \item Aerial silks, Music, Reading \& Traveling
    \end{itemize}
  
 
%\cvevent{ “Avaliação de atributos físicos dos sedimentos da Plataforma Noroeste Australiana”.}{Bachelor Thesis }{ 2019}{ UFF}
%\begin{itemize}
 %   \item Assessment of physical attributes of the Northwest % Shelf sediments” translated from the Portuguese
%\end{itemize}

%\cvevent{ANÁLISE DE SINAIS CICLOESTRATIGRÁFICOS E IDENTIFICAÇÃO DE QUASE-PERIODICIDADES EM ESCALA MILENAR PARA PROXIES DO SÍTIO U1463 (EXPEDIÇÃO IODP 356)}{Simpósio Brasileiro de Geologia e Geofísica Marinha}{2018}{Rio de Janeiro}
%\begin{itemize}
 %   \item CYLOSTRATIGRAPHIC SIGNALS ANALYSIS AND IDENTIFICATION OF NEAR PERIODICITIES ON A MILLENARY SCALE FOR PROXIES OF THE U1463 SITE (IODP EXPEDITION 356)
%\end{itemize}



